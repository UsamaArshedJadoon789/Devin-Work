\documentclass{article}
\usepackage[utf8]{inputenc}
\usepackage{listings}
\usepackage{graphicx}
\usepackage{hyperref}

\title{SIMP Processor Simulator Documentation}
\author{Computer Organization Project}
\date{\today}

\begin{document}
\maketitle

\section{Architecture Overview}
The SIMP (Simple Instruction Set Processor) is a 16-register processor with a 48-bit instruction format. Key features include:
\begin{itemize}
    \item 21 instructions including arithmetic, logical, memory, and control operations
    \item 4096-line instruction memory (48-bit)
    \item 4096-line data memory (32-bit)
    \item Three interrupt sources (timer, disk, external)
    \item I/O devices: LEDs, 7-segment display, monitor, disk
\end{itemize}

\section{Implementation Details}
\subsection{Core Components}
\begin{itemize}
    \item CPU: Implements instruction fetch, decode, and execute cycle
    \item Memory: Separate instruction and data memory management
    \item I/O: Device control and interrupt handling
    \item File I/O: Handles all input/output file operations
\end{itemize}

\subsection{Interrupt Handling}
\begin{itemize}
    \item Timer interrupt (IRQ0): Highest priority
    \item Disk interrupt (IRQ1): Medium priority
    \item External interrupt (IRQ2): Lowest priority
    \item No nested interrupts supported
\end{itemize}

\section{Test Programs}
\subsection{Matrix Multiplication (mulmat.asm)}
Implements multiplication of two 4x4 matrices:
\begin{itemize}
    \item First matrix: addresses 0x100-0x10F
    \item Second matrix: addresses 0x110-0x11F
    \item Result matrix: addresses 0x120-0x12F
\end{itemize}

\subsection{Binomial Coefficient (binom.asm)}
Recursive implementation of binomial coefficient calculation:
\begin{itemize}
    \item Input n: address 0x100
    \item Input k: address 0x101
    \item Result: address 0x102
    \item Uses stack for recursive calls
\end{itemize}

\subsection{Circle Drawing (circle.asm)}
Draws a filled white circle on the monitor:
\begin{itemize}
    \item Radius: address 0x100
    \item Center: (128,128)
    \item Uses distance formula for pixel selection
\end{itemize}

\subsection{Disk Test (disktest.asm)}
Tests disk operations by moving sector contents:
\begin{itemize}
    \item Moves contents of sectors 0-7 forward
    \item Handles disk busy status
    \item Uses interrupt-driven I/O
\end{itemize}

\section{File Format Specifications}
\subsection{Input Files}
\begin{itemize}
    \item imemin.txt: 12 hex digits per line (48-bit instructions)
    \item dmemin.txt: 8 hex digits per line (32-bit data)
    \item diskin.txt: 2 hex digits per line (8-bit disk data)
    \item irq2in.txt: Decimal cycle numbers for IRQ2
\end{itemize}

\subsection{Output Files}
\begin{itemize}
    \item dmemout.txt: 8 hex digits per line
    \item regout.txt: 8 hex digits per line (R2-R15)
    \item trace.txt: PC, instruction, and register values
    \item hwregtrace.txt: I/O register access log
    \item cycles.txt: Total cycle count
    \item leds.txt: LED states
    \item display7seg.txt: 7-segment display values
    \item diskout.txt: Final disk contents
    \item monitor.txt: Monitor pixel values
    \item monitor.yuv: Raw monitor output
\end{itemize}

\end{document}
